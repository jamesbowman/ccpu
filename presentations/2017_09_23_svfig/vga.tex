\documentclass[compress]{beamer}

\usetheme{naked}
\usecolortheme{dark}

\usepackage{pgfpages}
\setbeamertemplate{note page}[plain]
\setbeamercolor{note page}{fg=black}
\setbeameroption{show notes on second screen}
%\setbeameroption{hide notes} % Only slides

% \usepackage[utf8]{inputenc} % utf8 file encoding
% \usepackage[UTF8]{ctex}

\usepackage{listings}
\usepackage[T1]{fontenc} % powerful pdf output encoding

\title{Bit-banging VGA with the GA144}
\date{\today}
\author{James Bowman}

\begin{document}

\titlepage

\emptyslide

\begin{imageframe}{connector}
\note{
Even though it is from 1987, VGA is not quite dead.

Plenty of monitors have VGA connectors today.

It is an easy way to make a display -- just 5 signals.
}
\end{imageframe}

\begin{frame}
640x480 at 60 Hz

Pixel frequency is 25.175 MHz
\note{
tinyvga.com has a really useful list of VGA timings.

\url{http://tinyvga.com/vga-timing/640x480@60Hz}

A VGA 640x480 picture is clocked at 25 MHz, so each pixel takes 40
ns.
This is enough time for GA144 to execute 8-32 instructions.
}
\end{frame}
\begin{imageframe}{DSC_3223.JPG}
\note{
More details and links are at
\url{http://www.excamera.com/sphinx/article-ga144-vga.html}

Source is at

\url{https://github.com/jamesbowman/ga144tools/blob/master/src/vga.ga}
}
\end{imageframe}


\begin{imageframe}{layout}
\note{This is the layout}
\end{imageframe}

\begin{imageframe}{DSC_3236.JPG}
\note{This is RAMPS}
\end{imageframe}

\begin{imageframe}{DSC_3237.JPG}
\note{CHECKER}
\end{imageframe}

\begin{imageframe}{DSC_3238.JPG}
\note{CIRCLES}
\end{imageframe}

\begin{imageframe}{DSC_3239.JPG}
\note{CHECKER}
\end{imageframe}


\emptyslide

\end{document}
